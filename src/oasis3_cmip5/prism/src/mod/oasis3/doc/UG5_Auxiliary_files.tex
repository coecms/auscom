\newpage
\chapter{OASIS3 auxiliary data files}
\label{sec_auxiliary}

OASIS3 needs auxiliary data files describing coupling and I/O field
names and units, defining the grids of the models being coupled,
containing the field coupling restart values or input data values, as
well as a number of other auxiliary data files used in specific
transformations. 

\section{Field names and units}
\label{subsec_cfnametable}

The text file \texttt{cf\_name\_table.txt}, that can be found in
directory \texttt{prism/util/running\break/toyclim/input}
% XXX \texttt{prism/util/running\break/adjunct\_files/oasis3}
directory, contains a list of CF standard names and associated units
identified with an index. The appropriate index has to be given by the
user for each coupling or I/O field as the third entry on the field
first line (see \ref{subsec_namcouplesecond}). This information will
be used by OASIS3 for its log messages to {\it cplout} file and by the
PSMILe to produce CF compliant NetCDF files.

\section{Grid data files}
\label{subsec_griddata}

The grids of the models being coupled must be given by the user, or
directly by the model through PSMILe specific calls (see section
\ref{subsubsec_griddef}) in grid data files. These files can be all
binary or all NetCDF. In \break {\tt
/prism/data/toyclim/input\_toyclim\_standard\_standard\_prism\_2-2.tar.gz},
Net\-CDF examples can be found.

The
arrays containing the grid information are dimensioned {\tt (nx, ny)},
where {\tt nx} and {\tt ny} are the grid first and second dimension,
except for Unstructured (U) and Reduced (D) grid, for which the
arrays are dimensioned {\tt (nbr\_pts,1)}, where {\tt nbr\_pts} is the
total number of grid points. 

\begin{enumerate}

\item {\em grids} or {\em grids.nc}: contains the model grid
  longitudes, latitudes, and local angles (if any) in single or double
  precision {\tt REAL} arrays (depending on OASIS3 compilation
  options). The array names must be composed of a prefix (4
  characters), given by the user in the {\it namcouple} on the second
  line of each field (see section \ref{subsec_namcouplesecond}), and
  of a suffix (4 characters); this suffix is ``.lon'' or ``.lat'' for
  respectively the grid point longitudes or latitudes (see {\tt
    /prism/src/mod/oasis3/src\break/mod\_label.F90}.)

  For {\tt SCRIPR} interpolations, the grid data files must be NetCDF
  files. If the {\tt SCRIPR/CONSERV} remapping is used, longitudes and
  latitudes for the source and target grid {\bf corners} must also be
  available in the {\em grids.nc} file as arrays dimensioned {\tt
    (nx,ny,4)} or {\tt (nbr\_pts,1,4)} where {\tt 4} is the number
  of corners (in the counterclockwize sense). The names of the arrays
  must be composed of the grid prefix and the suffix ``.clo'' or
  ``.cla'' for respectively the grid corner longitudes or latitudes.
  As for the other grid information, the corners can be provided in
  {\em grids.nc} before the run by the user or directly by the model
  through PSMILe specific calls (see section \ref{subsubsec_griddef});
  furthermore, for Logically Rectangular LR source grids only, the
  grid corners will be automatically calculated if they are not
  available in {\em grids.nc} (if needed, the
  corresponding reverse remapping can be done in which the current
  target grid become the source grid).
 
 Longitudes must be given in degrees East in the interval -360.0 to
 720.0. Latitudes must be given in degrees North in the interval -90.0
 to 90.0. Note that if some grid points overlap, it is recommended to
 define those points with the same number (e.g. 360.0 for both, not
 450.0 for one and 90.0 for the other) to ensure automatic detection
 of overlap by OASIS. Note also that cells larger than
 180.0 degrees in longitude are not supported.

 If vector fields are defined on a grid which has a local coordinate
 system not oriented in the usual zonal and meridional directions, the
 local angle of the grid coordinate system must be given in {\it
   grids.nc} file in an array which name must be composed of the grid
 prefix and the suffix ``.ang''. The angle is defined as the angle
 between the first component and the zonal direction (which is also
 the angle between the second component and the meridional direction).
 In the grid file in {\tt
   /prism/data/toyclim/input\_toyclim\_standard\_standard\_prism\_2-2.tar.gz},
 the angles of the {\tt torc} grid are given in array {\tt torc.ang}.
 If one of the {\tt SCRIPR} interpolations is requested for a vector
 field, OASIS3 automatically performs the rotation from the local
 coordinate system to the geographic spherical coordinate system for a
 source grid, or vice-versa for a target grid.
 
 File {\em grids} or {\em grids.nc} must be present with at least the grid
 point longitudes and latitudes for all component model. 

\item {\em masks} or {\em masks.nc}: contains the masks for all
component model grids in {\tt INTEGER} arrays (0 -not masked- or 1
-masked- for each grid point). The array names must be composed of the
grid prefix and the suffix ``.msk''. This file, {\em masks} or {\em
  masks.nc}, is mandatory.

\item {\em areas} or {\em areas.nc}: this file contains mesh surfaces
for the component model grids in single or double precision {\tt REAL}
arrays (depending on OASIS3 compilation options). The array names must be
composed of the grid prefix and the suffix ``.srf''.  The surfaces may
be given in any units but they must be all the same (in {\tt
INTERP/GAUSSIAN}, it is assumed that the units are $m^2$ but they are
used for statistics calculations only.) This file {\em areas} or {\em
areas.nc} is mandatory for {\tt CHECKIN}, {\tt CHECKOUT} or {\tt
CONSERV}, and used for statistic calculations in {\tt
INTERP/GAUSSIAN}; it is not required otherwise.

\item {\em maskr} or {\em maskr.nc}: this file
  contains Reduced (D) grid mask in {\tt INTEGER} arrays dimensioned
  {\tt array(nbr\_pts)} where {\tt nbr\_pts} is the total number of the
  Reduced grid points (0 -not masked- or 1 -masked- for each grid
  point). This file is required only for grids to which the {\tt
  REDGLO} or {\tt GLORED} transformation is applied. {\it As mentionned above,
  these transformations should not be used anymore as interpolations
  are now available for Reduced grids directly.} If used, the mask
  array name must be ``MSKRDxxx'' where ``xxx'' is half the number of
  latitude circles of the reduced grid (032 for a T42 for example).
  
\end{enumerate}

If the binary format is used, {\em grids}, {\em masks}, {\em areas},
and {\em maskr} must have the following structure. The array
name is first written to
the file to locate a data set corresponding to a given grid.  The
data set is then written sequentially after its name.  Let us call
``brick'' the name and its associated data set.  The order in which
the bricks are written doesn't matter.  All the bricks are written in
the grid data file in the following way:

\begin{verbatim}
        ...
        WRITE(LU) array_name
        WRITE(LU) auxildata
        ...\end{verbatim} where
\begin{itemize}
\item {\tt LU} is the associated unit,
\item {\tt array\_name} is the name of the array (CHARACTER*8),
\item {\tt auxildata} is the REAL or INTEGER array dimensioned {\tt
  (nx, ny)} or {\tt (nbr\_pts,1)} containing the grid data.
\end{itemize}

%subsection{Grid data files}

\section{Coupling restart files}
\label{subsec_restartdata}

At the beginning of a coupled run, some coupling fields may have to be
initially read from their coupling restart file (see section \ref
{subsubsec_Algoritms}). If needed, these files are also automatically
updated by the last {\tt prism\_put\_proto} call of the run (see
section \ref{prismput}) .  To force the writing of the field in its
coupling restart file, one can use the routine {\tt
  prism\_put\_restart\_proto} (see section \ref{subsec:auxiliary}).
{\bf Warning}: the date is not written or read to/from the restart file;
therefore, the user has to make sure that the appropriate restart file
is present in the working directory.

The name of the coupling restart file is
given by the 6th character string on the first configuring line for
each field in the {\em namcouple} (see section
\ref{subsec_namcouplesecond}). Coupling fields coming from different
models cannot be in the same coupling restart files, but for each
model, there can be an arbitrary number of fields written in one
coupling restart file. Note that in the NONE techniques, output files
with the same format are also created for writing the resulting field
after transformation.

In the coupling restart files, the fields must be single or double
precision REAL arrays (depending on PSMILe and OASIS3 compilation
options) and, as the grid data files, must be dimensioned {\tt (nx,
ny)}, where {\tt nx} and {\tt ny} are the grid first and second
dimension, except for fields given on Unstructured ('U') and Reduced
('D�) grid, for which the arrays are dimensioned {\tt (nbr\_pts,1)},
where {\tt nbr\_pts} is the total number of grid points.  The shape
and orientation of each restart field (and of the corresponding
coupling fields exchanged during the simulation) must be coherent with
the shape of its grid data arrays. 
%The exceptions are for A, B, G,
%L, Z, or Y grids for which the field may be oriented from North to
%South and/or from East to West, in which case, {\tt INVERT}
%transformation will have to be used - see section
%\ref{subsec_preproc}.

Both binary and NetCDF formats are supported; for NetCDF file the
suffix .nc is not mandatory. If the coupling restart file
for the first field is in NetCDF format, OASIS3 will assume that all
coupling restart files (and output files for {\tt NONE}
communication techniques) are NetCDF\footnote{Note that even if the grid
auxiliary data files are in NetCDF format, the restart coupling files
may be in binary format, or vice-versa.}. 

In the NetCDF restart files, the field arrays must have the source symbolic
name indicated in the {\em namcouple}
(see section \ref{subsec_namcouplesecond}).

In binary restart file, each field is written in the following way:
\begin{verbatim}
        ...
        WRITE(LU) array_name
        WRITE(LU) restartdata
        ...\end{verbatim} where
\begin{itemize}
\item {\tt LU} is the associated unit,
\item {\tt array\_name} is the source symbolic name of the field (CHARACTER*8),
\item {\tt restartdata} is the restart field REAL array dimensioned
  {\tt (nx, ny)} or {\tt (nbr\_pts,1)}\footnote{If REDGLO is the first
  transformation applied on a Reduced grid field, the Reduced field
  must be given is an array {\tt restartdata(nx*ny)} where {\tt nx}
  and {\tt ny} are the global Gaussian grid dimensions and the Reduced
  field is completed by trailing zeros.}
\end{itemize}

%subsection{Restart data files}

\section{Input data files}
\label{subsec_inputdata}

Fields with status {\tt INPUT} in the {\em namcouple} will, at
  runtime, simply be read in from a NetCDF input file by the target
  model PSMILe below the {\tt prism\_get\_proto} call, at appropriate
  times corresponding to the input period indicated by the user in the
  {\it namcouple}. 

\vspace*{0.5cm}
 
The name of the file must be the one given on the field first
configuring line in the {\em namcouple} (see section
\ref{subsubsec_secondINPUT}). There must be one input file per {\tt
INPUT} field, containing a time sequence of the field in a single or
double precision REAL array (depending on PSMILe compilation options),
named with the field symbolic name in the {\em namcouple} and
dimensioned {\tt (nx,ny,time)} or {\tt (nbr\_pts,1,time)}.  The time
variable as to be an array {\tt time(time)} expressed in
``seconds since beginning of run''. The ``time'' dimension has to
be the unlimited dimension. 
For a practical example, see the file SOALBEDO.nc in 
{\tt /prism/data/toyclim/input\_toyclim\_standard\_standard\_prism\_2-2.tar.gz}.

%subsection{Input data files}

\section{Transformation auxiliary data files}
\label{subsec_transformationdata}

Many transformation need auxiliary data files, such as the
grid-mapping files used for an interpolation. Some of them are created
automatically by OASIS3, others have to be generated by the user before
starting the coupled run.

\subsection{Auxiliary data files for {\tt EXTRAP/NINENN}, 
{\tt EXTRAP/WEIGHT}, {\tt INTERP/SURFMESH}, {\tt INTERP/GAUSSIAN},
{\tt MOZAIC}, and {\tt SUBGRID}}

The auxiliary data files containing the weights and addresses used
in these transformations have a similar structure; their names are
given in Table \ref{tab.fileana}.

\begin{table}[hbtp]
\begin{center}
\begin{tabular}{|l|l|}
\hline
File name & Description \\
\hline
\hline
{\em  nweights }& weights, addresses and iteration number for
EXTRAP/NINENN interpolation  \\
any name        & weights and addresses for EXTRAP/WEIGHT extrapolation \\
{\em  mweights }& weights and addresses for INTERP/SURFMESH interpolation  \\
{\em  gweights }& weights and addresses for INTERP/GAUSSIAN interpolation  \\
any name        & weights and addresses for MOZAIC interpolation \\

any name        & weights and addresses for SUBGRID interpolation \\
\hline
\end{tabular}
\end{center}
\caption{Analysis auxiliary data files}
\label{tab.fileana}
\end{table}

The files {\em nweights}, {\em mweights} and {\em gweights} can be
created by OASIS3 if their corresponding {\tt \$NIO} = 1 (see {\tt
EXTRAP/NINENN, INTERP/SURFMESH, INTERP/GAUSSIAN} in sections
\ref{subsec_preproc} and \ref{subsec_interp}).

The name of the (sub)grid-mapping files for {\tt MOZAIC,
EXTRAP/WEIGHT} and {\tt SUBGRID} analyses can be chosen by the user
and have to be indicated in the {\em namcouple} (see respectively
sections \ref{subsec_preproc} and \ref{subsec_interp} and
\ref{subsec_cooking}). These files have to be generated by the user before
starting the coupled run.

\vspace*{0.5cm}
 
The structure of these files is as follows:

\begin{verbatim}
      ...
      CHARACTER*8 cladress,clweight
      INTEGER iadress(jpnb,jpo)
      REAL weight(jpnb,jpo)
      OPEN(unit=90, file='at31topa', form='unformatted')
      WRITE(clweight,'(''WEIGHTS'',I1)') knumb
      WRITE(cladress,'(''ADRESSE'',I1)') knumb
      WRITE (90) clweight
      WRITE (90) weight
      WRITE (90) cladress
      WRITE (90) iadress
\end{verbatim}

where 
\begin{itemize}
\item {\tt jpnb} is the maximum number of neighbors used in the transformation
({\tt \$NVOISIN} in the {\em namcouple})
\item {\tt jpo} is the total dimension of the target grid
\item {\tt at31topa} is the name of the grid-mapping data file ({\tt \$CFILE}
in  {\em namcouple})
\item {\tt knumb} is the identificator of the data set ({\tt \$NID} in
{\em namcouple}) 
\item {\tt cladress} is the locator of the address dataset
\item {\tt clweight} is the locator of the weight dataset
\item {\tt iadress (i,j)} is the address on the source grid of the $i^e$
neighbor used for the mapping of the $j^e$ target grid point. The
address is the index of a grid point within the total number of grid
points.
\item {\tt weight(i,j)} is the weight affected to the $i^e$
neighbor used for the transformation of the $j^e$ target grid point
\end{itemize}

For file {\em nweights}, there is an additional brick composed of a
{\tt CHARACTER*8} variable (formed by the characters {\tt INCREME} and
by the data set identificator) and of an {\tt INTEGER array(N)} which
is the iteration number within {\tt EXTRAP/NINENN} at which the
extrapolation of the $n^e$ grid point is effectively performed.

\subsection{Auxiliary data files for {\tt FILLING}}

For the FILLING analysis, the global data set used can be
either interannual monthly, climatological monthly or yearly (see
\ref{subsec_interp}). The name of the global data file 
can be chosen by the user and has to be indicated in the {\em namcouple}
have to be given to OASIS through the input file {\em namcouple}. In case of monthly data, the file 
must be written in the following way:
\begin{verbatim}
      ...
      REAL field_january_year_01(jpi, jpj)
      ...
      WRITE(NLU_fil) field_january_year_01
      WRITE(NLU_fil) field_february_year_01
      WRITE(NLU_fil) field_march_year_01
      etc...
      WRITE(NLU_fil) field_december_year_01
C
C if climatology, one stops here
C
      WRITE(NLU_fil) field_january_year_02
      etc...
\end{verbatim}

where
\begin{itemize}
\item  {\tt field\_...} is the global dataset
\item {\tt jpi} and {\tt jpj} are the dimensions of the grid on which FILLING
is performed
\item {\tt NLU\_fil} is the logical unit associated to the global data file and is
defined in the input file {\em namcouple}
\end{itemize}
Note that the first month needs not to be a january.
This is the only file within OASIS in which the fields are not read
using a locator.

\subsection{Auxiliary data files for {\tt SCRIPR}}

The NetCDF files containing the weights and addresses for the {\tt
  SCRIPR} remappings (see section \ref{subsec_interp})  are
  automatically generated at runtime by OASIS3. Their structure is
  described in detail in section 2.2.3 of the SCRIP documentation available
  in {\tt prism/src/mod/oasis3/doc/SCRIPusers.pdf}.

