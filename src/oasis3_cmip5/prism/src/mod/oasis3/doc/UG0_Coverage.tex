\newpage
\thispagestyle{empty}
%\begin{figure}
%\includegraphics[scale=.8]{cover_2-3.eps} 
%\label{cover}
%\end{figure}
\vspace{50mm}


%\eject
%\clearpage
%\thispagestyle{empty}
%.

%\newpage

\vspace*{0.5cm}
{\Large
   \begin{center}
      OASIS3\\
      {\bf O}cean {\bf A}tmosphere {\bf S}ea  {\bf I}ce  {\bf S}oil\\
      \vspace{0.4cm}
      User's Guide \\
      \vspace{0.4cm}
      {\large oasis3\_prism\_2-3, August 2004}

   \end{center}
}

\vspace*{1cm}

\centerline{Sophie Valcke$^1$, Arnaud Caubel$^2$, Reiner
  Vogelsang$^3$, Damien Declat$^1$}

\vspace*{1cm}
\centerline{PRISM Project WP3a}
\centerline{$^1$C.E.R.F.A.C.S.} 
\centerline{$^2$FECIT/Fujitsu}
\centerline{$^3$SGI Germany}

\begin{abstract}
This User's Guide contains a detailed step-by-step description on 
how to realize a coupled simulation with OASIS3.

The aim of OASIS3 is to provide a flexible and user friendly tool 
for coupling independent
general circulation models of the atmosphere and the ocean (A/O-GCMs) 
as well as other climate component models (sea-ice, land,
atmospheric chemistry, ocean biogeochemistry, ...) and regional models. 
The resulting coupled models are necessary tools to tackle
current climatic paradigms such as the natural variability, 
El Ni\~no Southern Oscillation
(ENSO) or the greenhouse gas global warming effect. 

OASIS3 synchronizes the exchanges of coupling fields between 
the models being coupled, and performs 
2D interpolations and transformations needed to express, on the grid
of the target model, the coupling
fields produced by the source model on its grid. Modularity and
flexibility have been particularly emphasized in OASIS3 design.

\end{abstract}
