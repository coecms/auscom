\newpage
\chapter{The OASIS3 configuration file {\it namcouple}}
\label{sec_namcouple}

The OASIS3 configuration file {\em namcouple} contains, below
pre-defined keywords, all user's
defined information necessary to configure a particular coupled
run. The {\it namcouple} is a text file with the following
characteristics:

\begin{itemize}
\item the keywords used to separate the information
can appear in any order;
\item the number of blanks between two character strings is
non-significant; 
\item all lines beginning with \# are ignored and considered as
comments.
\item {\bf blank lines are not allowed.}
\end{itemize}
\vspace*{0.5cm}

The first part of {\em namcouple } is devoted to configuration of
general parameters such as the number of models involved in the
simulation, the number of fields, the communication technique, etc.
The second part gathers specific information on each coupling or I/O
field, e.g. their coupling period, the list of transformations or
interpolations to be performed by OASIS3 and their associated
configuring lines (described in more details in section
\ref{sec_transformations}), etc.

\vspace*{0.5cm}

In the next sections, a simple {\it namcouple} example is given and
all configuring parameters are described. The additional lines
containing the different parameters required for each transformation
are described in section \ref{sec_transformations}. An example of a
realistic {\em namcouple} can be found in directory 
{\tt
/prism/util/running\break/toyclim/input/namcouple}.
% XXX and {\tt
%/prism/util/running\break/adjunct\_files/oasis3/namcouple\_toyclim\_use}.

\section{An example of a simple {\it namcouple}}
\label{subsec_examplenamcouple}

The following simple {\it namcouple} configures a run in which an
ocean, an atmosphere and an atmospheric chemistry models are
coupled. The ocean provides only the SOSSTSST field to the atmosphere,
which in return provides the field CONSFTOT to the ocean. One field
(COSENHFL) is exchanged directly from the atmosphere to the
atmospheric chemistry, and one field (SOALBEDO) is read from a file by
the ocean.

\begin{verbatim}

######################################################################
# First section
#
 $SEQMODE
    1
#
 $CHANNEL
    MPI2    NOBSEND
    1    1   arg1
    3    1   arg2
    3    1   arg3            
#
 $NFIELDS
    4
#
 $JOBNAME
    JOB
#
 $NBMODEL
    3  ocemod   atmmod  chemod  55  70   99 
#
 $RUNTIME
    432000
#
 $INIDATE
    00010101
#
 $MODINFO
    NOT
#
 $NLOGPRT
   2
#
 $CALTYPE
   1
#
######################################################################
# Second section 
#
 $STRINGS
#
# Field 1
#
 SOSSTSST SISUTESU 1 86400  5  sstoc.nc  EXPORTED
 182  149  128  64  toce  atmo   LAG=+14400  SEQ=+1
 P 2 P 0
 LOCTRANS CHECKIN MOZAIC  BLASNEW CHECKOUT 
#
  AVERAGE 
  INT=1
  at31topa   91    2   48
  CONSTANT     273.15 
  INT=1
#
# Field 2
#
 CONSFTOT SOHEFLDO 6 86400  7   flxat.nc  EXPORTED
 atmo   toce  LAG=+14400  SEQ=+1
 P 0 P 2
 LOCTRANS  CHECKIN  SCRIPR CHECKOUT
#
  ACCUMUL 
  INT=1
  CONSERV  LR  SCALAR  LATLON   10  FRACAREA   FIRST
  INT=1
#
# Field 
#
 COSENHFL  SOSENHFL  37  86400   1  flda3.nc  IGNOUT 
 atmo   atmo LAG=+7200  SEQ=+1
 LOCTRANS
 AVERAGE
#
# Field 4
#
 SOALBEDO SOALBEDO  17  86400  0  SOALBEDO.nc  INPUT
#
#####################################################################
\end{verbatim}

%section{An example of a simple {\em namcouple}}

\section{ First section of {\it namcouple} file}
\label{subsec_namcouplefirst}

The first section of {\it namcouple } uses some predefined keywords
prefixed by the \$ sign to locate the related information. The
\$ sign must be in the second column. The first ten keywords
are described hereafter:

\begin{itemize}

\item {\tt \$SEQMODE}: On the line below this keyword is the maximum number of fields that
  have to be, at one particular coupling timestep,
  necessarily exchanged sequentially in a given order.  For {\tt
  \$SEQMODE} $\ge 1$, the position of each coupling field in the sequence has to be
  given by its SEQ index (see below and also section
  \ref{subsubsec_Algoritms}).

\item {\tt \$CHANNEL}: On the line below this keyword is the
communication technique chosen. Choices are {\tt MPI1} or {\tt MPI2}
for the CLIM communication technique and related PSMILe library, using
{\tt MPI1} or {\tt MPI2} message passing.  To run OASIS3 as an
interpolator only, put {\tt NONE} (see also section
\ref{subsec_interpolator}). 
The communication techniques available in previous OASIS version, i.e.
{\tt SIPC}, {\tt PIPE}, or {\tt GMEM} should still work but are not
officially supported anymore and were not tested.

To use the {\tt CLIM/MPI2} communication technique, the lines below {\tt
  \$CHANNEL} are, e.g. for 3 models:
\begin{verbatim} 
$CHANNEL 
MPI2 NOBSEND
1    1  arg1 
3    1  arg2
3    1  arg3 
\end{verbatim}
where {\tt MPI2} is the message passing used in CLIM and PSMILe, and
{\tt NOBSEND} indicates that standard blocking send {\tt MPI\_Send}
should be used in place of the buffered {\tt MPI\_BSend} to send the
coupling fields.\footnote{Use the standard blocking send {\tt
MPI\_Send} if the coupling fields are necessarily sent and received in
the same order, or on platforms for which {\tt MPI\_Send} is
implemented with a mailbox (e.g. VPPs; in this case, make sure
that the size of the mailbox is sufficient). Use the less efficient
buffered send {\tt MPI\_BSend} on platforms for which {\tt
MPI\_Send} is not implemented with a mailbox if the
coupling fields are not sent and received in the same order.
Note that below the call to {\tt prism\_enddef\_proto}, the PSMILe
tests whether or not the model has already attached to an MPI
buffer. If it is the case, the PSMILe detaches from the buffer, adds
the size of the pre-attached buffer to the size needed for the
coupling exchanges, and reattaches to an MPI buffer. The model own
call to {\tt MPI\_Buffer\_Attach} must therefore be done before the
call to {\tt prism\_enddef\_proto}. Furthermore, the model is not
allowed to call {\tt MPI\_BSend} after the call to {\tt
prism\_terminate\_proto}, as the PSMILe definitively detaches from the
MPI buffer in this routine. See the example in the toyatm model in
{\tt prism/src/mod/toyatm/src} }

If {\tt NOBSEND} is not specified, the buffered send {\tt MPI\_BSend}
will be used.

The following lines (one line per model listed on the {\tt \$NBMODEL}
line) indicate for each model the total number of processes, the
number of processes implied in the coupling, and possibly launching
arguments. Here the first model runs on one process which is of course
implied in the coupling and the argument passed to the model is
"arg1"; the second and third models run on 3 processes but only one
process is implied in the coupling (i.e. exchanging information with
OASIS3 main process), and the argument passed to the models are
respectively ``arg2" and ``arg3''.

To use the {\tt CLIM/MPI1} communication technique, the {\tt
  \$CHANNEL} lines are are as for {\tt MPI2} except that {\tt MPI2} is
replaced by {\tt MPI1} and there is no launching
arguments\footnote{With MPI1, models have to be started by the user in
  a pseudo-MPMD mode in the order they are introduced in the {\it
    namcouple}. The way to do this depends on the computing platform.
  With MPI1, OASIS3 main process and the component models
  automatically share the same MPI\_COMM\_WORLD communicator; in this
  communicator OASIS3 main process is assumed to have rank 0 and the
  other component models are assumed to have ranks in the order of
  which they are introduced in namcouple. If this is not the case, a
  deadlock may occur.}.

\item {\tt \$NFIELDS}: On the line below this keyword is the total
number of fields exchanged and described in the second part of
the {\it namcouple}.

\item {\tt \$JOBNAME}: On the line below this keyword is a {\tt
CHARACTER$\star$3} or {\tt CHARACTER$\star$4} variable giving an
acronym for the given simulation.

\item {\tt \$NBMODEL}: On the line below this keyword is the number of
models running in the given experiment followed by {\tt
CHARACTER$\star$6} variables giving their names. Then the user may
indicate the maximum Fortran unit number used by the models. In the
example, Fortran units above 55, 70, and 99 are free for respectively
the ocean, atmosphere, and atmospheric chemistry models. If no
maximum unit numbers are indicated, OASIS3 will suppose that units
above 1024 are free. If {\tt \$CHANNEL}
is {\tt NONE}, {\tt \$NBMODEL} has to be 0 and there should be no
model name and no unit number.

\item {\tt \$RUNTIME}: On the line below this keyword is the total
simulated time of the run, expressed in seconds. If {\tt \$CHANNEL}
is {\tt NONE}, {\tt \$RUNTIME} has to be the number of time occurrences
of the field to interpolate from the restart file. 

\item {\tt \$INIDATE}: On the line below this keyword is the initial
  date of the run. The format is {\tt YYYYMMDD}. This date is important only
  for the {\tt FILLING} transformation and for printing information in
  OASIS3 log file {\it cplout}.

\item {\tt \$MODINFO}: If coupling restart files are binary files (see
  section \ref{subsec_restartdata}), the line below this keyword
  indicates if a header is encapsulated or not: it can be {\tt
  YES} or {\tt NOT}.
 
\item {\tt \$NLOGPRT}: The line below this keyword refers to the
  amount of information that will be written to the OASIS3 log file
  {\it cplout} during the run. With 0, there is practically no output
  written to the cplout; with 1, only some general information on the
  run, the header of the main routines, and the names of the fields
  when treated appear in the {\it cplout}.  Finally, with 2, the full output
  is generated.

\item {\tt \$CALTYPE}: This new keyword gives the type of calendar
  used. For now, the calendar type is important only if {\tt FILLING}
  analysis is used for a coupling field in the run and for printing
  information in OASIS3 log file {\it cplout}. Below this keyword, a
  number (0, 1 or n ) must be indicated by the user:
  \begin{itemize}
  \item 0 : a 365 day calendar (no leap year) 
  \item 1 : a 365 or 366 (leap years) day calendar A year is a leap
  year if it can be divided by 4; however if it can be divided by 4
  and 100, it is not a leap year; furthermore, if it can be divided by
  4, 100 and 400, it is a leap year.
  \item n :  $n \ge 1$ day month calendar. 
  \end{itemize}

\end{itemize}

%{Description of {\it namcouple} first section}

\section{Second section of {\it namcouple} file }
\label{subsec_namcouplesecond}

The second part of the {\it namcouple}, starting after the keyword
{\tt \$STRINGS}, contains coupling information for each coupling or
I/O field.  Its format depends on the field status given by the last
entry on the field first line ({\tt EXPORTED}, {\tt IGNOUT} or {\tt
INPUT} in the example above). The field status may be the following
({\tt AUXILARY} and {\tt EXPORTED} are supported by all communication
techniques, while the others are supported only by the PSMILe i.e. the
{\tt CLIM/MPI1} or {\tt CLIM/MPI2} communication technique):

\begin{itemize}
\item {\tt AUXILARY}: sent by the source model, received and used by
  OASIS3 main process for the transformation of other fields.
\item {\tt EXPORTED}: exchanged between component models and
  transformed by OASIS3 main process.
\item {\tt EXPOUT}: exchanged, transformed and also written to two
  output files, one before the sending action in the source model
  below the {\tt prism\_put\_proto} call, and one after the receiving
  action in the target model below the {\tt prism\_get\_proto} call.
\item {\tt IGNORED}: exchanged directly between the component models
  without being transformed by OASIS3 main process. The grid and
  partitioning of the source and target models have to be
  identical.
\item {\tt IGNOUT}: exchanged directly between the component models
  without being transformed by OASIS3 main process and written to two
  output files, one before the sending action in the source model
  below the {\tt prism\_put\_proto} call, and one after the receiving
  action in the target model below the {\tt prism\_get\_proto} call.
  The grid and partitioning of the source and target models have to be
  identical.
\item {\tt INPUT}: simply read in from the input file by the target
  model PSMILe below the {\tt prism\_get\_proto} call at appropriate
  times corresponding to the input period indicated by the user in the
  {\it namcouple}. See section
  \ref{subsec_inputdata} for the format of the input file.
\item {\tt OUTPUT}: simply written out to an output file by the source
  model PSMILe below the {\tt prism\_put\_proto} call at appropriate
  times corresponding to the output period indicated by the user in
  the {\it namcouple}. The name of the output file (one per field) is
  automatically built based on the field name and initial date of the
  run ({\tt \$INIDATE}).

\end{itemize}

\subsection{Second section of {\it namcouple} for {\tt EXPORTED}, {\tt
      AUXILARY} and {\tt EXPOUT} fields}
\label{subsubsec_secondEXPORTED}

  The first 3 lines for fields with status {\tt EXPORTED}, {\tt
  AUXILARY} and {\tt EXPOUT} are as follows:
  \begin{verbatim}
   SOSSTSST SISUTESU 1 86400  5  sstoc.nc  sstat.nc EXPORTED
   182  149    128  64  toce  atmo   LAG=+14400 SEQ=+1
   P 2 P 0 
  \end{verbatim}
  where the different entries are:
    \begin{itemize}
      \item Field first line:
        \begin{itemize}
        \item {\tt SOSSTSST} : symbolic name for the field in the
              source model ({\tt CHARACTER*8}). It has to match the
              argument {\tt name} of the corresponding field
              declaration in the source model; see {\tt
              prism\_def\_var\_proto} in section
              \ref{subsubsec_Declaration}.
        \item {\tt SISUTESU} : symbolic name for the field in the
              target model ({\tt CHARACTER*8}).  It has to match the
              argument {\tt name} of the corresponding field
              declaration in the target model; see {\tt
              prism\_def\_var\_proto} in section
              \ref{subsubsec_Declaration}.
        \item 1 : index in auxiliary file cf\_name\_table.txt used by OASIS3 and PSMILe to identify corresponding CF standard name and units (see \ref{subsec_cfnametable}).
        \item 86400 : coupling and/or I/O period for the field, in
        seconds. (If {\tt \$CHANNEL} is {\tt NONE}, put ``1''.)
        \item 5 : number of transformations to be performed on this field.  
        \item sstoc.nc : name of the coupling restart file for the
  	  field ({\tt CHARACTER*8}); it may be a binary of netCDF file (for
	  more detail, see section \ref{subsec_restartdata}).
        \item sstat.nc : name of the field output file, may be indicated for
	  {\tt NONE} (and {\tt PIPE}) communication techniques only. It
	  may be a binary of netCDF file (see section
	  \ref{subsec_restartdata}).
        \item {\tt EXPORTED} : field status.
        \end{itemize}
      \item Field second line:
        \begin{itemize}
        \item 182 : number of points for the source grid first
        dimension (optional if a netCDF coupling restart file is used). 
        \item 149 : number of points for the source grid second
        dimension (optional if a netCDF coupling restart file is used).    
        \item 128 : number of points for the target grid first
        dimension (optional if a netCDF coupling restart file is used). 
        \item 64 : number of points for the target grid second
        dimension (optional if a netCDF coupling restart file is used).  
        \item toce : prefix of the source grid name in grid data files (see section
        \ref{subsec_griddata}) ({\tt CHARACTER*4})
        \item atmo : prefix of the target grid name in grid data files
        ({\tt CHARACTER*4})
        \item {\tt LAG=+14400}: optional lag index for the field
        expressed in seconds ({\tt CLIM/MPI1} or {\tt CLIM/MPI2}
        communication technique only, see section
        \ref{subsubsec_Algoritms}). Note that in mode NONE a LAG has
        to be defined so that the input file is opened initially.
        \item {\tt SEQ=+1}: optional sequence index for the field
        ({\tt CLIM/MPI1} or {\tt CLIM/MPI2} communication technique
        only, see section \ref{subsubsec_Algoritms}).
        \end{itemize}
      \item Field third line
        \begin{itemize}
         \item P : source grid first dimension characteristic
            (`P': periodical; `R': regional).
         \item 2 : source grid first dimension number of overlapping grid points.
         \item P : target grid first dimension characteristic (`P':
         periodical; `R': regional).
         \item 0 : target grid first dimension number of overlapping grid points.
        \end{itemize}
     
      \end{itemize}
    
  The fourth line gives the list of transformations to be performed
  for this field. There is then one or more additional configuring lines
  describing some parameters for each transformation. These
  additional lines are described in more details in the section
  \ref{sec_transformations}.

\subsection{Second section of {\it namcouple} for {\tt
  IGNORED}, {\tt IGNOUT}, and {\tt OUTPUT} fields}
\label{subsubsec_secondIGNORED}
  The first 2 lines for fields with status {\tt IGNORED} or {\tt
  IGNOUT} or {\tt OUTPUT} are as follows:
  \begin{verbatim}
  COSENHFL SOSENHFL 37 86400 1 flda3.nc IGNOUT 
  atmo   toce LAG=+7200 SEQ=+1\end{verbatim}where the different
  entries are as for {\tt EXPORTED} fields, except that there is no
  output file name on the first line.

  For {\tt OUTPUT} fields, there is no target model and therefore no
  target symbolic name; the source symbolic name must be repeated
  twice on the field first line. Also, there is no coupling restart
  file name ({\tt flda3.nc} here), no {\tt LAG} index and no {\tt SEQ}
  index.

  For {\tt IGNORED} fields, the name used in the coupling restart file
  (if any) must be the target symbolic name.

\vspace*{0.5cm}

  The third line is {\tt LOCTRANS} if this transformation is chosen
  for the field. Note that {\tt LOCTRANS} is the only transformation
  supported for {\tt IGNORED}, {\tt IGNOUT} and {\tt OUTPUT} fields
  (as it is performed directly in the PSMILe below the {\tt
  prism\_put\_proto} call). If {\tt LOCTRANS} is chosen, a fourth line
  giving the name of the time transformation is required. For more
  detail on {\tt LOCTRANS}, see section \ref{subsec_timetrans}.

\subsection{Second section of {\it namcouple} for {\tt
  INPUT} fields}
\label{subsubsec_secondINPUT}

  The first and only line for fields with status {\tt INPUT} is:

  \begin{verbatim}
  SOALBEDO SOALBEDO  17  86400  0  SOALBEDO.nc  INPUT\end{verbatim} where the different entries are:
  \begin{itemize}
  \item  {\tt SOALBEDO}: symbolic name for the field in the target
  model ({\tt CHARACTER*8} repeated twice)
  \item 17:  index in auxiliary file cf\_name\_table.txt (see above for EXPORTED fields)
  \item 86400: input period in seconds
  \item 0: number of transformations (always 0 for {\tt INPUT} fields)
  \item {\tt SOALBEDO.nc}: {\tt CHARACTER*32} giving the input file
  name (for more detail on its format, see section
  \ref{subsec_inputdata})
  \item {\tt INPUT}: field status.
  \end{itemize}

