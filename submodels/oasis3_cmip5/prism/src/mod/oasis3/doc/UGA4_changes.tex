\newpage

\chapter{Changes between versions}
\label{sec_changes}

Here is a list of changes between the different official OASIS3
versions.
\section{Changes between {\tt oasis3\_prism\_2\_5} and {\tt
oasis3\_prism\_2\_4}}

The changes between version {\tt oasis3\_prism\_2\_5} and version {\tt
  oasis3\_prism\_2\_4} delivered in December 2004 are listed here
after. Please note that those modifications should not bring any
difference in the interpolation results, except for SCRIPR/DISTWGT
(see below).

\begin{itemize}

\item Bug corrections:

  \begin{itemize}

  \item In {\tt prism/src/lib/scrip/src/scriprmp.F}: initialisation of
    {\tt dst\_array(:)}; bug fix announced to the mailing list
    diff-oasis@cerfacs.fr on 02/02/2006.

  \item In {\tt prism/src/lib/psmile/src/prism\_enddef\_proto.F} and {\tt
      prism/src/lib/\break clim/src/CLIM\_Start\_MPI.F}: the call to
    MPI\_barrier (that created a deadlock when not all processes of a
    component model were exchanging data with the coupler) was changed
    for a call to MPI\_wait on the previous MPI\_Isend; bug fix
    announced to the mailing list diff-oasis@cerfacs.fr on
    02/23/2006.

  \item For SCRIPR/DISTWGT, in {\tt
      prism/src/lib/scrip/src/remap\_distwgt.f}: line 190 was repeated
    without epsilon modification; bug fix announced to the mailing
    list diff-oasis@cerfacs.fr on 03/21/2006.

  \item In {\tt prism/src/lib/psmile/src/mod\_prism\_put\_proto.F90},
    for {\tt prism\_put\break \_proto\_r28} and {\tt
      prism\_put\_proto\_r24}, the reshape of the 2d field was moved
    after the test checking if the field is defined in the namcouple
    (thanks to Arnaud Caubel from LSCE).

  \end{itemize}

\item Modification in SCRIP interpolations

  \begin{itemize}

  \item For {\tt SCRIPR} interpolations (see section
    \ref{subsec_interp}), the value 1.0E+20 is assigned to target grid
    points for which no value has been calculated if {\tt
      prism/src/lib/scrip/src/scriprmp.f} or {\tt vector.F90} (for
    vector interpolation) are compiled with {\tt ll\_weightot =
      .true.}.

  \item For {\tt SCRIPR/GAUSWGT}: if routine {\tt
      prism/src/lib/scrip/src/remap\_gauswgt.f} is compiled with {\tt
      ll\_nnei=.true.}, the non-masked nearest neighbour is used for
    target point if all original neighbours are masked (see section
    \ref{subsec_interp}).

  \item For {\tt SCRIPR/BICUBIC} (routine {\tt
      prism/src/lib/scrip/src/remap\_bicubic.f}), the convergence
    criteria was modified so to ensure convergence even in single
    precision.

  \item For {\tt SCRIPR/CONSERV} (routine {\tt
      prism/src/lib/scrip/src/remap\_conserv.f}), a test was added for
    non-convex cell so that integration does not stall.

  \item The routine {\tt prism/src/lib/scrip/src/corners.F} was
    modified so to abort if it is called for the target grid, as the
    automatic calculation of corners works only for
    Logically-Rectangular (LR) grids and as the target grid type is
    unknown. If needed, the reverse remapping, in which the current
    target grid become the source grid, can be done .

  \end{itemize}

\item Other important modifications

  \begin{itemize}

  \item A new PSMILe routine {\tt
      prism/src/lib/psmile/src/prism\_get\_freq.F} was added; this
    routine can be used to retrieve the coupling period of field (see
    section \ref{subsec:auxiliary}).

  \item The routines of the {\tt mpp\_io} library in {\tt
      prism/src/lib/mpp\_io} changed name and were merged with the
    OASIS4 {\tt mpp\_io} library.

  \item Routine {\tt prism/src/mod/oasis3/src/extrap.F} was modified
    to ensure that the extrapolation works even if the {\tt MASK}
    value is very big (thanks to J.M. Epitalon).

  \item In the namcouple, there is no need anymore to define a lag
    greater than 0 (e.g.  LAG=+1) for fields in mode NONE.

  \item Diverse modifications were included for successful compilation
    with NAGW compiler: non portable use of ``kind'', etc. (thanks to
    Luis Kornblueh from MPI).

  \item In {\tt prism/src/lib/psmile/mod\_prism\_get\_proto.F90}, a
    potential deadlock was removed (the master process was sending a
    message to itself)(thanks to Luis Kornblueh from MPI).

  \item Routine {\tt prism/src/lib/scrip/src/scriprmp\_vector.F90} was
    completely rewritten for more clarity.

  \item Obsolete transformations INVERT and REVERSE were removed from
    the toy coupled model TOYCLIM (in file {\tt
      prism/util/running/toyclim/input/namcouple}. This change does
    not affect the statistics printed in the {\tt cplout} but changes
    the orientation of some fields in the NetCDF ouput files (see the
    results in {\tt prism/data/toyclim/outdata}).

\end{itemize}

\item Other minor modifications:

  \begin{itemize}

  \item In {\tt prism/src/lib/psmile/src/prism\_enddef\_proto.F},
    allocation is done only for rg\_field\_trans or dg\_field\_trans
    depending on precision for REAL (but not for both, to save
    memory).

  \item In few routines in {\tt prism/src/lib/clim} and in {\tt
      prism/src/mod/oasis3}, parentheses were added to make sure that
    \&\& has priority over $|$$|$ in CPP instructions (thanks to A.
    Caubel from LSCE).

%\item corners.f: changed intent(INOUT) grid\_center\_lon/lat 
% for intent(IN) src\_lon/lat and added case for ymean==0
%\item In inipar.F, modification so that restart file
%number and restart file name arrays are not filled up if the field
%does not have a restart file.

  \item Routines {\tt scrip/src/corners.f}, {\tt netcdf.f}, and {\tt
      scriprmp.f} were renamed \newline {\tt corners.F}, {\tt netcdf.F}, {\tt
      scriprmp.F} and the line ``INCLUDE 'netcdf.inc' '' was changed
    for ``\#include $<$netcdf.inc$>$ ''

  \end{itemize}

\end{itemize}

\section{Changes between {\tt oasis3\_prism\_2\_4} and {\tt
      oasis3\_prism\_2\_3}}

  The changes between versions tagged {\tt oasis3\_prism\_2\_4} and {\tt
    oasis3\_prism\_2\_3} delivered in July 2004 are the following:

\begin{itemize}

\item Update of compiling and running environments with version
  \texttt{prism\_2-4} of PRISM Standard Compiling Environment (SCE) and
  PRISM Standard Running Environment (SRE), which among other
  improvements include the environments to compile and run on the CRAY
  X1 (see the directories with \texttt{<node>=baltic1}), thanks to
  Charles Henriet from CRAY France, and on a Linux station from
  Recherche en Pr\'evision Num\'erique (Environnement Canada, Dorval,
  Canada) (see the directories with \texttt{<node>=armc28}).

\item \texttt{prism/src/mod/oasis3/src/iniiof.F}: the opening of the
  coupling restart files is done only if the corresponding field has a
  lag greater than 0; note that this implies that all fields in mode
  NONE must now have a lag greater than 0 (e.g. LAG=+1) (thanks to
  Veronika Gayler from M\&D).

\item \texttt{prism/src/lib/psmile/src/prism\_def\_var\_proto.F}:
 contrary to what was previously described in the documentation,
 \texttt{PRISM\_Double} is not supported as $7^{th}$ argument to
 describe the field type; {\tt PRISM\_Real} must be given for single
 or double precision real arrays.

\item \texttt{prism/src/mod/oasis3/src/inipar.F90}: For upward
   compatibility of SCRIPR interpolation, ``VECTOR" is still accepted
   in the namcouple as the field type 
   and leads to the same behaviour as before (i.e. each vector
   component is treated as an independent scalar field). To have a
   real vector treatment, one has to indicate "VECTOR\_I" or "VECTOR\_J"
   (see section \ref{subsec_interp}).

\item Bug corrections in: 
\begin{itemize}

\item \texttt{prism/src/lib/scrip/src/scriprmp\_vector.F90}: In some
 cases, some local variables were not deallocated and variable
 \texttt{dimid} was declared twice.

\item 
\texttt{prism/src/lib/psmile/src/mod\_psmile\_io.F90}: correct
allocation of array hosting the longitudes (thanks to Reiner Vogelsang
from SGI Germany).

\item
\texttt{prism/src/lib/psmile/src/write\_file.F90}: to remove a deadlock
on some architecture (thanks to Luis Kornblueh from MPI).

\item
\texttt{prism/src/lib/psmile/src/prism\_enddef\_proto.F}: the error
handler is now explicitely set to \texttt{MPI\_ERRORS\_RETURN} before
the call to \texttt{MPI\_Buffer\_Detach} to avoid abort on some
architecture when the component model is not previously attached to
any buffer (thanks to Luis Kornblueh from MPI). 



\item 
  \texttt{prism/src/lib/scrip/src/remap\_conserv.f} (thanks to Veronika
  Gayler from M\&D).

\item \texttt{prism/src/mod/oasis3/src/inicmc.F}

\item \texttt{prism/src/lib/scrip/src/remap\_distwgt.f}

\end{itemize}

\end{itemize}

\section{Changes between {\tt oasis3\_prism\_2\_3} and {\tt
oasis3\_prism\_2\_2}}

The changes between versions tagged {\tt oasis3\_prism\_2\_3}
delivered in July 2004 and {\tt oasis3\_prism\_2\_2} delivered
in June 2004 are the following:

\begin{itemize}

\item Bug correction of the previous bug fix regarding ordering of grid and
data information contained in I/O files when {\tt INVERT} or {\tt
REVERSE} transformations are used: the re-ordering now occurs only for
source field if {\tt INVERT} is used, and only for target field if
{\tt REVERSE} is used.

\item LGPL license: OASIS3 is now officially released under a Lesser GNU General Public
License (LGPL) as published by the Free Software Foundation
(see {\tt prism/src/mod/oasis3/COPYRIGHT} and {\tt prism/src/mod/oasis3/src/couple.f})

\item Upgrade of compiling and running environments: The compiling and
running environments have been upgraded to the PRISM Standard
Compiling and Running Environment version dated August 5th
2004, that should be very close to ``prism\_2-3''.

\item Treament of vector fields: The interpolation algorithms using the SCRIP library now support
vector fields, including automatic rotation from local to geographic
coordinate system, projection in Cartesian coordinate system and
interpolation of 3 Cartesian components, and support of vector
components given on different grids. New routines have been added in
{\tt prism/src/lib/scrip/src}: {\tt scriprmp\_vector.F90} and {\tt rotations.F90}.
For more detail, see {\tt SCRIPR} in section \ref{subsec_interp}.

\item All include of mpif.h are now written `\#include $<$mpif.h$>$'.

\item The output format of {\tt CHECKIN} and {\tt CHECKOUT} results
is now E22.7

\end{itemize}

\section{Changes between {\tt oasis3\_prism\_2\_2} and {\tt
oasis3\_prism\_2\_1}}

The changes between versions tagged {\tt oasis3\_prism\_2\_2}
delivered in June 2004 and {\tt oasis3\_prism\_2\_1} delivered to
PRISM in April 2004 are the following:

\begin{itemize}

\item Bug corrections

 \begin{itemize}

 \item {\tt INTERP/GAUSSIAN} and {\tt SCRIPR/GAUSWGT} transformations
 work for `U' grids.

 \item The grid and data information contained in I/O files output by
 the PSMILe library have now a coherent ordering even if {\tt INVERT}
 or {\tt REVERSE} transformations are used.

 \end{itemize}

\item OASIS3 and the TOYCLIM coupled model are ported to IBM Power4 and
Linux Opteron, which are now included in the Standard Compiling and
Running Environments (SCE and SRE).

\item SIPC technique communication is re-validated.

\item {\tt Clim\_MaxSegments = 338} in {\tt
prism/src/lib/clim/src/mod\_clim.F90} and in {\tt
prism/src/lib/psmile/src/mod\_prism\_proto.F90}. 338 is presently the largest
value needed by a PRISM model. 

\item {\tt MPI\_BSend}: below the call to {\tt prism\_enddef\_proto},
the PSMILe tests whether or not the model has already attached to an
MPI buffer. If it is the case, the PSMILe detaches from the buffer,
adds the size of the pre-attached buffer to the size needed for the
coupling exchanges, and reattaches to an MPI buffer. The model own
call to {\tt MPI\_Buffer\_Attach} must therefore be done before the
call to {\tt prism\_enddef\_proto}. Furthermore, the model is not
allowed to call {\tt MPI\_BSend} after the call to {\tt
prism\_terminate\_proto}, as the PSMILe definitively detaches from the
MPI buffer in this routine. See the example in the toyatm model in
{\tt prism/src/mod/toyatm/src}.

\end{itemize}

\section{Changes between {\tt oasis3\_prism\_2\_1} and {\tt
oasis3\_prism\_1\_2}}

The changes between versions tagged {\tt oasis3\_prism\_1\_2}
delivered in September 2003 and {\tt oasis3\_prism\_2\_1} delivered to
PRISM in April 2004 are the following:

\begin{itemize}

\item Bug corrections

 \begin{itemize}

 \item Thanks to Eric Maisonnave, a bug was found and corrected in
  \- \- \- \- \- \- \- \- 
  /prism/src/lib/scrip/src/scriprmp.f:
  ``sou\_mask'' and ``tgt\_mask'' were not properly initialised if weights
  and addresses were not calculated but read from file.

 \item Some deallocation were missing in prism\_terminate\_proto.F
  (``ig\_def\_part'', ``ig\_length\_part'', ``cg\_ignout\_field'').

 \item Thanks to Arnaud Caubel, a bug was found and corrected in 
  \- \- \- \- \- \- \- \- /prism/src/lib/psmile/src/write\_file.F90. In case of parallel
  communication between a model and OASIS3 main process, the binary
  coupling restart files were not written properly (NetCDF coupling
  restart files are OK).

 \end{itemize} 

\item Routines renamed

The routines {\tt preproc.f, extrap.f, iniiof.f} in {\tt
prism/src/mod/oasis3/src} were renamed to {\tt preproc.F, extrap.F,
iniiof.F}, as a CPP key `key\_openmp' was added. Please note that this
key, allowing openMP parallelisation, is not fully tested yet.

\item Modifications in the namcouple

\begin{itemize}
\item The third entry on the field first line now corresponds to an index in
the new auxiliary file {\em cf\_name\_table.txt} (see sections
\ref{subsec_namcouplesecond} and \ref{subsec_cfnametable}).

\item For {\tt IGNORED, IGNOUT} and {\tt OUTPUT} fields, the source
and target grid locator prefixes must now be given on the field second
line (see section \ref{subsubsec_secondIGNORED})
\end{itemize}

\item A new auxiliary file {\em cf\_name\_table.txt}

For each field, the CF standard name used in the OASIS3 log file, {\em
cplout}, is now defined in an additional auxiliary file {\em
cf\_name\_table.txt} 
% (see the file in \- {\tt prism/data/toyclim/input})XXXX, 
not in {\tt inipar.F} anymore. This auxiliary file must be
copied to the working directory at the beginning of the run. The user
may edit and modify this file at her own risk. In {\em
cf\_name\_table.txt}, an index is given for each field standard name
and associated units.  The appropriate index has to be indicated for
each field in the {\em namcouple} (third entry on the field first
line, see section \ref{subsec_namcouplesecond}). 

This standard name and the associated units are also used to define the
field attributes ``long\_name'' and ``units'' in the NetCDF output files
written by the PSMILe for fields with status {\tt EXPOUT, IGNOUT} and
{\tt OUTPUT}.
 
For more details on this auxiliary file, see section
\ref{subsec_cfnametable}.

\item Many timesteps for mode NONE

In mode NONE, OASIS3 can now interpolate at once all time occurrences
of a field contained in an input NetCDF file. The time variable in the
input file is recognized by its attribute ``units''. The acceptable
units for time are listed in the udunits.dat file \cite{udunits}. This
follows the CF convention.

The keyword {\tt \$RUNTIME} in the namcouple has to be the number of time
occurrences of the field to interpolate from the input file. The
``coupling'' period of the field (4th entry on the field first line)
must be always ``1''. Note that if {\tt \$RUNTIME} is smaller than the
total number of time ocurrences in the input file, the first {\tt \$RUNTIME} 
occurrences will be interpolated.

For more details, see section \ref{subsec_interpolator}.

\item Model grid data file writing

The grid data files {\em grids.nc, masks.nc} and {\em areas.nc} can
now be written directly at run time by the component models, if they
call the new routines prism\_start\_grids\_writing, prism\_write\_grid,
prism\_write\_corner prism\_write\_mask, prism\_write\_area,
prism\_terminate\_grids\_writing. 

The writing of those grid files by the models is driven by the
coupler. It first checks whether the binary file {\em grids} or the
netCDF file {\em grids.nc} exists (in that case, it is assumed that
{\em areas} or {\em areas.nc} and {\em masks} or {\em masks.nc} files
exist too) or if writing is needed. If {\em grids} or {\em grids.nc}
exists, it must contain all grid information from all models; if it does not
exist, each model must write its grid informations in the grid data
files. 

See section \ref{subsubsec_griddef} for more details.

\item Output of CF compliant files

The NetCDF output files written by the PSMILe for fields with status
{\tt EXPOUT, IGNOUT} and {\tt OUTPUT} are now fully CF compliant.

In the NetCDF file, the field attributes ``long\_name'' and ``units''
are the ones corresponding to the field index in {\em
cf\_name\_table.txt} (see above and section
\ref{subsec_cfnametable}). The field index must be given by the user
as the third entry on the field first line in the namcouple.

Also, the latitudes and the longitudes of the fields are now
automatically read from the grid auxiliary data file {\em grids.nc} and
written to the output files. If the latitudes and the longitudes of
the mesh corners are present in {\em grids.nc}, they are also written to the
ouput files as associated ``bounds'' variable.  This works whether the
{\em grids.nc} is given initially by the user or written at run time by the
component models (see above). However, this does not work if the
user gives the grid definition in a binary file {\em grids}.

\item Removal of pre-compiling key ``key\_BSend''


The pre\_compiling key ``key\_BSend'' has been removed.  The default has
changed: by default, the buffered MPI\_BSend is used, unless {\tt NOBSEND}
is specified in the namcouple after MPI1 or MPI2, in which case the
standard blocking send MPI\_Send is used to send the coupling fields.

\end{itemize}



