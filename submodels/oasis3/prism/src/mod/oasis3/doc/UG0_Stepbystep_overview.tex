\newpage
\chapter{Introduction}
\label{sec_step}

OASIS3 is the direct evolution of the OASIS coupler developed since
more than 10 years at CERFACS (Toulouse, France).  OASIS3 is a
portable set of Fortran 77, Fortran 90 and C routines.  At run-time,
OASIS3 acts as a separate mono process executable, which main function
is to interpolate the coupling fields exchanged between the component
models, and as a library linked to the component models, the OASIS3
PRISM Model Interface Library (OASIS3 PSMILe). OASIS3 supports 2D
coupling fields only. To communicate with OASIS3, directly with
another model, or to perform I/O actions, a component model needs to
include few specific PSMILe calls. OASIS3 PSMILe supports in
particular parallel communication between a parallel component model
and OASIS3 main process based on Message Passing Interface (MPI) and
file I/O using the mpp\_io library from GFDL.  Portability and
flexibility are OASIS3 key design concepts.  OASIS3 has been
extensively used in the PRISM demonstration runs and is currently used
by approximately 15 climate modelling groups in Europe, USA, Canada,
Australia, India and Brasil.  The current OASIS3 version and its toy
coupled model TOYCLIM were compiled and run on NEC SX6,
IBM Power4 and Linux PC cluster, and previous OASIS3 versions were run
on many other platforms.

\section{Step-by-step use of OASIS3}

To use OASIS3 for coupling models and/or perform I/O
actions, one has to follow these steps:
\begin{enumerate}
\item Obtain OASIS3 sources. (See chapter \ref{sec_Obtaining}).
\item Identify the coupling or I/O fields and adapt the component
  models to allow their exchange with the PSMILe library based on MPI1
  or MPI2 message passing\footnote{The SIPC, PIPE and GMEM
    communication techniques available in previous versions should
    still work but are not maintained anymore and were not tested.}.
  The PSMILe library is interfaced with the {\tt mpp\_io} library from
  GFDL \cite{mpp_io} and therefore can be used to perform I/O actions
  from/to disk files.  For more detail on how to interface a model
  with the PSMILe, see chapter \ref{sec_modelinterfacing}.

The TOYCLIM coupled model gives a practical example of a coupled
model; the sources are given in directories {\tt
  /prism/src/mod/toyatm, /toyoce, /toyche}; more detail on TOYCLIM and
how to compile and run it can be found in chapter
\ref{sec_compilationrunning}.

\item Define all coupling and I/O parameters and the transformations
  required to adapt each coupling field from its source model grid to
  its target model grid; prepare OASIS3 configuring file {\it
  namcouple} (See chapter \ref{sec_namcouple}). 
  OASIS3 supports different interpolation algorithms as is described in
  chapter \ref{sec_transformations}.

\item Generate required auxiliary data files. (See chapter
  \ref{sec_auxiliary}).
\item Compile OASIS3, the component models and start the coupled
  experiment. Chapter \ref{sec_compilationrunning} describes how to
  compile and run OASIS3 and the TOYCLIM coupled model.

\end{enumerate}

The appendix \ref{sec_couplings} lists (some of) the coupled
models realized with OASIS within the past 5 years or so.  

If you need extra help, do not hesitate to contact us (see contact
details on the back of the cover page).

