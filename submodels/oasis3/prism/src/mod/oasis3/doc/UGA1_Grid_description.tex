\newpage
\appendix
\chapter{The grid types for the transformations}
\label{subsec_gridtypes}

As described in section \ref{sec_transformations}, the different
transformations in OASIS3 support different types of grids. The
characteristics of these grids are detailed here.

\begin{enumerate}

\item Grids supported for the {\tt INTERP}
    interpolations (see section \ref{subsec_interp})

\begin{itemize}
 
\item {\tt `A' grid}: this is a regular Lat-Lon grid covering either
      the whole globe or an hemisphere, going from South to North and
      from West to East.  There is no grid point at the
      pole and at the equator, and the first latitude has an offset of
      0.5 grid interval. The first longitude is 0$^o$ (the Greenwhich
      meridian) and is not repeated at the end of the grid ({\tt
      \$CPER} = P and {\tt \$NPER}= 0).  The latitudinal grid length
      is 180/NJ for a global grid, 90/NJ otherwise. The longitudinal
      grid length is 360/NI. 

\item {\tt `B' grid}: this is a regular Lat-Lon grid covering either
      an hemisphere or the whole globe, going from South to North and
      from West to East. There is a grid point at the
      pole and at the equator (if the grid is hemispheric or global
      with NJ odd). The first longitude is 0$^o$ (the Greenwhich
      meridian), and is repeated at the end of the grid ({\tt \$CPER}
      = P and {\tt \$NPER}= 1).  The latitudinal grid length is
      180/(NJ-1) for a global grid, 90/(NJ-1) otherwise. The
      longitudinal grid length is 360/(NI-1). 

  
\item {\tt `G' grid}: this is a irregular Lat-Lon Gaussian grid
covering either an hemisphere or the whole globe, going from South to
North and from West to East. This grid is used in spectral models. It
is very much alike the A grid, except that the latitudes are not
equidistant. There is no grid point at the pole and at the
equator. The first longitude is 0$^o$ (the Greenwhich meridian) and is
not repeated at the end of the grid ({\tt \$CPER} = P and {\tt
\$NPER}= 0).  The longitudinal grid length is 360/NI.  


\item {\tt `L' grid}: this type covers regular Lat-Lon grids in
      general, going from South to
North and from West to East.. The grid can be described by the latitude and the
      longitude of the southwest corner of the grid, and by the
      latitudinal and longitudinal grid mesh sizes in degrees.

\item {\tt `Z' grid}: this is a Lat-Lon grid with non-constant
      latitudinal and longitudinal grid mesh sizes, going from South to
North and from West to East. The deformation of
      the mesh can be described with the help of 1-dimensional
      positional records in each direction. This grid is periodical
      ({\tt \$CPER} = P) with {\tt \$NPER} overlapping grid points.

\item {\tt `Y' grid}: this grid is like `Z' grid except that it is 
      regional ({\tt \$CPER} = R and {\tt \$NPER} = 0).
  
 \end{itemize}

\item Grids supported for the {\tt SCRIPR} interpolations

\begin{itemize}

\item {\tt `LR' grid}: The longitudes and the latitudes of
  2D Logically-Rectangular (LR) grid points can be described by two arrays
  {\tt longitude(i,j)} and {\tt latitude(i,j)}, where i and j
  are respectively the first and second index dimensions. Streched
  or/and rotated grids are LR grids. Note that A, B, G, L, Y, or Z
  grids are all particular cases of LR grids.

\item {\tt `U' grid}: Unstructured (U) grids do have any particular
      structure. The longitudes and the latitudes of 2D Unstructured
      grid points must be described by two arrays {\tt
      longitude(nbr\_pts,1)} and {\tt latitude(nbr\_pts,1)}, where nbr\_pts
      is the total grid size.

\item {\tt `D' grid} The Reduced (D) grid is composed of a certain
number of latitude circles, each one being divided into a varying
number of longitudinal segments. In OASIS3, the grid data (longitudes,
latitudes, etc.) must be described by arrays dimensioned {\tt
(nbr\_pts,1)}, where {\tt nbr\_pts} is the total number of grid
points. There is no overlap of the grid, and no grid point at the
equator nor at the poles. There are grid points on the Greenwich
meridian.
 
\end{itemize}

\end{enumerate}



