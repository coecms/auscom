%
%$Id: observationsIntro.tex,v 16.0 2008/07/30 22:40:32 fms Exp $
%

\section{Working with observed data in GOTM}

In the context of GOTM, the term `observations' should be understood
in a broad sense: it may refer to data either measured in nature or
generated artificially.  The inclusion of such data into GOTM can
serve different purposes. Examples are time-series of external
pressure-gradients, which can be used to drive the model, or observed
profiles of the temperature to which model results can be relaxed.

Two different types of `observations' are considered so far in GOTM:
time series of scalar data and time series of profile data. The first
type is used to introduce, for example, sea surface elevations into
the model. The latter is used to include, for example, temperature or
velocity fields.


All specifications concerning the `observations' are done via the
namelist file {\tt obs.inp}.  Each of type of variable has its own
namelist in {\tt obs.inp}.  Common for all namelists is a member with
the suffix {\tt
\_method}, used to specify the action performed to generate or
acquire the variable, respectively. Observations can be, for example,
read-in from files or computed according to an analytical expression.
Some types of observations (e.g.\ turbulent dissipation rates) are not
used directly during the calculations in GOTM. but can be conveniently 
interpolated to the numerical grid to allow for an easy comparison
of measured data and model results.

For all types of observations, one {\tt \_method } is always `from
file'.  All input-files are in ASCII with a very straight-forward
format. The necessary interpolation in space is performed as an
integral part of the general reading routines. Temporal interpolation
is performed as part of the specific reading routines, e.g.\ {\tt
get\_s\_profile.F90}.

